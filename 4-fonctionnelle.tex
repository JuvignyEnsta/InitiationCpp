\documentclass[handout,10pt]{beamer}
\usetheme{Copenhagen}
\usepackage{amsmath,amsfonts,graphicx}
\usepackage{listings}
\usepackage{colortbl}
\usepackage{enumitem}
\setitemize{label=\usebeamerfont*{itemize item}%
            \usebeamercolor[fg]{itemize item}
            \usebeamertemplate{itemize item}}
            
\setlist[description]{style = multiline, labelwidth = 60pt}

\usepackage{tikz} % Required for drawing custom shapes
\usetikzlibrary{shadows, arrows, decorations.pathmorphing, fadings, shapes.arrows, positioning, calc, shapes, fit, matrix}

\usepackage{polyglossia}


\definecolor{lightblue}{RGB}{0,200,255} 
\definecolor{paper}{RGB}{245,241,207}
\definecolor{ocre}{RGB}{243,102,25} % Define the orange color used for highlighting throughout the book
\definecolor{BurntOrange}{RGB}{238,154,0}
\definecolor{OliveGreen}{RGB}{188,238,104}
\definecolor{DarkGreen}{RGB}{0,128,0}
\definecolor{BrickRed}{RGB}{238,44,44}
\definecolor{Tan}{RGB}{210,180,140}
\definecolor{Aquamarine}{RGB}{127,255,212}
\definecolor{NavyBlue}{RGB}{0,64,128}

\title[C++\hspace{2em}]{Programmation fonctionnelle avec C++ 2014}
\author[Xavier JUVIGNY]{Xavier JUVIGNY}
\date{22 Juillet 2017}

\institute{ONERA}

\begin{document}

\lstset{language=C++,
  frame=single,
  backgroundcolor=\color{paper},
  basicstyle=\tiny\ttfamily,
  keywordstyle=\color{blue}\tiny\ttfamily,
  commentstyle=\color{red}\tiny\ttfamily,
  stringstyle=\color{brown}\tiny\ttfamily,
  keepspaces=true,
  showspaces=false,
  tabsize=4
}

\begin{frame}
 \titlepage
\end{frame}

\begin{frame}
\frametitle{Plan du cours}
\tableofcontents
\end{frame}

\section{Programmation fonctionnelle}

\begin{frame}[fragile]{Le C++ : un langage principalement orienté objet ?}
\tiny
\begin{block}{Cas de la STL}
\begin{itemize}
\item Beaucoup de classes;
\item Très peu de classes sont conçues pour être dérivées;
\item Très peu de fonctions virtuelles ( pures ou non );
\item Bien plus de POO dans \lstinline$iostream$...
\end{itemize}
\end{block}
\begin{alertblock}{Questions}
\begin{itemize}
\item Le commité de normalisation de C++ a-t'il oublié de concervoir la STL en orienté objet ?
\item Est-e que le C++ est un langage orienté objet ?
\begin{exampleblock}{Ben, ça dépend...}
\begin{itemize}
\item Oui si vous suggérez que le C++ supporte la programmation orienté objet;
\item Non si vous suggérez que c'est son principal paragdime de programmation.
\end{itemize}
\end{exampleblock}
\end{itemize}
\end{alertblock}
\end{frame}

\begin{frame}[fragile]{Problèmes liés avec la POO}
\begin{block}{Ne répond pas à des questions essentielles}
\begin{itemize}
 \item {\color{DarkGreen}Comment écrire de meilleurs algorithmes composables ?} :\\ La POO est basé sur le vieux principe
 de la programmation impérative;
 \item {\color{DarkGreen}Comment écrire du code rapide et de haut niveau ?} :\\ Les structures de données de la POO souvent
 trop lents à cause des indirections et de la perte d'info sur le type à la compilation ( type erasure );
 \item {\color{DarkGreen}Comment maîtriser la complexité dans des logiciels multithreadés ?} :\\ La POO ne dit rien à ce sujet. 
\end{itemize}
\alert{On a donc besoin d'un autre paradigme} :\\ la plupart des programmeurs C++ le savent déjà...
\end{block}
\end{frame}

\begin{frame}[fragile]{Qu'est ce que la programmation fonctionnelle ?}
\tiny
\begin{exampleblock}{Principes de base}
\begin{itemize}
\item Les fonctions sont les acteurs principaux;
\item On ne peut pas changer l'état des objets du programme;
\item Les fonctions sont pures, composables, génériques et reutilisables;
\item Fait un lourd usage du système de typage;
\item Exemple de langage fonctionnel : {\color{blue}Scheme ( 1975 ), Lisp ( 1984 ), Haskell (1987), Caml, \ldots}
\end{itemize}
\end{exampleblock}

\begin{block}{Et en C++ ?}
\begin{itemize}
 \item C++ n'est pas vraiment un langage fonctionnel;
 \item Mais beaucoup d'applications ont besoin d'être très rapides tout en gérant beaucoup de complexité;
 \item C++ est rapide et la programmation fonctionnelle aide à gérer la complexité.
 \item Les programmeurs C++ peuvent souvent bénéficier de fonctionnalités ( et de codage ) bien pensées;
 \item Le C++ 11 a introduit un nombre intéressant d'éléments de programmation fonctionnelle.
\end{itemize}
\end{block}

\end{frame}

\section{Programmer en fonctionnel avec C++ 14}

\begin{frame}[fragile]{Programmer dans un style fonctionnel}
\tiny
\begin{block}{Les services proposés par le C++}
\begin{itemize}
\item Déduction automatique de type avec \lstinline$auto$ et \lstinline$decltype$;
\item Support les \textsl{fonctions lambda} :
\item Application partielle fonctionnelle : \lstinline$std::function$ et \lstinline$std::bind$ + fonctions lambdas
\item Fonctions de plus grand ordre dans la STL;
\item \textsl{Manipulation de listes} au travers des \textbf{variadic templates};
\item Correspondance de motif à l'aide de la spécialisation partielle ou non des templates;
\item \'Evaluation paresseuse avec \lstinline$std::async$;
\item Template avec types contraints ( en utilisant \lstinline$static_assert$ ).
\end{itemize}
\end{block}
\end{frame}

\begin{frame}[fragile]{Déduction automatique de type}
\tiny
\begin{block}{Déduction automatique de type avec auto}
\begin{itemize}
 \item \lstinline$auto$ joue le rôle d'un paramètre template :
 \begin{lstlisting}
int x = 22;             // as before
const int cx = x;       // as before
const int& rx = x;      // as before
auto& v1 = x;           // v1's type ≡ int& (auto ≡ int)
auto& v2 = cx;          // v2's type ≡ const int&  (auto ≡ const int)
auto& v3 = rx;          // v3's type ≡ const int&  (auto ≡ const int)
const auto& v4 = x;     // v4's type ≡ const int&  (auto ≡ int)
const auto& v5 = cx;    // v5's type ≡ const int&  (auto ≡ const int)
const auto& v6 = rx;    // v6's type ≡ const int&  (auto ≡ const int)\end{lstlisting}
\end{itemize}
\end{block}

\begin{block}{Déduction des types par observation}
\begin{itemize}
 \item Utilisation du mot clef \lstinline$decltype$ $\equiv$ type déclaré pour nom
 \item Contrairement à \lstinline$auto$, ne supprime jamais les const/volatile/references.
\begin{lstlisting}
int x = 10;         // decltype(x)  ≡ int
const auto& rx = x; // decltype(rx) ≡ const int& 
int arr[10];
arr[0] = 5;         // decltype(arr[0]) ≡ int&
\end{lstlisting}
\item Mais si on met une variable en ``expression'' :
\begin{lstlisting}
int x; // decltype(x) ≡ int, decltype((x)) ≡ int&
\end{lstlisting}
\end{itemize}
\end{block}
\end{frame}

\begin{frame}[fragile]{Niveau d'abstraction des fonctions}
 \begin{block}{Qu'est ce donc ?}
  \begin{itemize}
   \item Une fonction d'ordre plus élevé est une fonction qui prend pour argument des fonctions et retourne
   d'autres fonctions;
   \item Les fonctions dans \lstinline$algorithm$ en sont de bon exemples
   \item L'introduction de fonctions lambdas dans C++ 11 permet de dire officiellement que cette approche est
   encouragée en C++.
  \end{itemize}
 \end{block}
\end{frame}

\begin{frame}[fragile]{Les fonctions lambdas}
\tiny
\begin{block}{Définition}
 \begin{itemize}
  \item Les fonctions lambdas sont des objets fonctions anonymes;
\begin{lstlisting}
// Trie par valeur absolue avec std::sort de algorithm :
std::sort(std::begin(v),std::end(v), [](double x, double y){ return std::abs(x)<std::abs(y); });
auto func = [] () { cout << "Hello world"; };
func();
\end{lstlisting}
  \item Capture de variables, plusieurs modes :
  \begin{itemize}
  \item \lstinline$[]$ Ne capture rien ( fonction standard )
  \item \lstinline$[&]$ Capture les variables visibles par référence
  \item \lstinline$[=]$ Capture les variables visibles en les copiant
  \item \lstinline$[=,&foo]$ : Capture les variables visibles en les copiant sauf \lstinline$foo$ capturée par référence
  \item \lstinline$[bar]$ : Ne capture que la variable \lstinline$bar$;
  \item \lstinline$[this]$ : Capture le pointeur sur l'objet courant.
  \end{itemize}
\begin{lstlisting}
double scal;
auto homothetie = [scal](double x){return scal*x;};
scal = 2; std::cout << homothetie(3.) << std::endl;// Affiche 6
scal = 4; std::cout << homothetie(3.) << std::endl;// Affiche 12
\end{lstlisting}
  \item Pour prendre un lambda en paramètre, utiliser un paramètre template :
\begin{lstlisting}
template<typename Iterator, typename F> void for_each(Iterator b, Iterator e, F f ) {
    for (; b != e; ++b ) f(*b);
}
\end{lstlisting}

 \end{itemize}
\end{block}
\end{frame}

\begin{frame}[fragile]{Propriétés des fonctions lambda}
\tiny
\begin{exampleblock}{Avantages}
\begin{itemize}
 \item N'introduisent pas de surcout supplémentaire en temps d'appel;
 \item Chaque lambda est mis dans son propre type anonyme;
 \item Le corps de la fonction est mis dans l'opérateur membre \lstinline$operator()$ du type;
 \item Les variables capturées sont mises comme attributs de la classes
 \item les lambdas jouent subtilement entre comment les templates marchent et comment les compilateurs font
 du inlining pour l'appel des fonctions.
\end{itemize}
\end{exampleblock}
\begin{alertblock}{Inconvénients en C++ 11}
\begin{itemize}
 \item Ne peuvent pas avoir d'arguments génériques;
 \item Ils ne peuvent pas capturer des variables par déplacement;
 \item Le type anonyme ne peut pas être nommé donc les lambdas ne peuvent pas être retournées d'une fonction;
 \item \alert{Une fonction lambda ne peut pas être utilisée dans une constexpr}.
\end{itemize}
\end{alertblock}
\begin{block}{Améliorations en C++ 14}
\begin{itemize}
 \item Les arguments peuvent être génériques : 
 \lstinline$auto f= [] ( auto& a, auto& b ) { return a*b; };$
 \item Déduction du type de retour automatique dans les fonctions standards : les lambdas peuvent être maintenant retournées
 par une fonction ( dont les fonctions lambda ! )
 \item Capture des listes généralisées;
\end{itemize}
\end{block}
\end{frame}

\begin{frame}[fragile]{Manipulation des fonctions lambdas}
\tiny
\begin{block}{\lstinline$std::function$}
\begin{itemize}
 \item Permet de passer une fonction lambda à une fonction non template ou bien de stocker une fonction lambda pour
 une utilisation ultérieure
\begin{lstlisting}
std::function<int(int)> f = [] (int x) { return x * 2; };
\end{lstlisting}
\item C'est un wrapper polymorphique autour de tout objet appelable;
\item Mis en {\oe}uvre autour de technique d'oublie de types;
\item Rajoute un surcoût de fait de l'appel indirect et du non inlining;
\item Ce surcoût est le même que pour tout autre langage supportant les fonctions lambdas;
\item Pour le passage d'une fonction en argument, préférer passer la fonction en template.
\end{itemize}
\end{block}
\begin{block}{Combinaison d'ordre plus élevé}
\begin{itemize}
 \item Style fonctionnel pour combiner des fonctions pour ordre plus élevé possible en C++
 \item Les ingrédient clefs : fonctions lambdas, type automatique, déduction, etc.
 \item C++ rend les choses un peu plus complexe : gérer le passage parfait aux suivants, réduire les opérations de copie/
 déplacement, etc.
 \item Beaucoup de librairies nous permettent d'avoir du fun sans trop de mal de tête : Boost Fusion, Boost Hana, Fit, range \ldots
\end{itemize}
\end{block}
\end{frame}

\begin{frame}[fragile]{Exemple : Composition de fonctions}
 \begin{block}{Ce qu'on aimerait avoir}
\begin{lstlisting}
auto f(std::vector<float>) -> float;
auto g(std::string)        -> std::vector<float>
auto h(std::istream&)      -> std::string

auto fgh = compose(f,g,h);
int x = fgh(std::cin); // x = f(g(h(std::cin)));
\end{lstlisting}
\end{block}

\begin{block}{Facile à faire en C++ 14}
\begin{lstlisting}
template<typename F> auto compose(F f) { return [=](auto x) { return f(x); } };
template<typename F, typename... Fs> auto compose(F f, Fs... fs) {
    return [=](auto x){ return f(compose(fs...)(x)); };    
}
\end{lstlisting}
\end{block}
\end{frame}

\begin{frame}[fragile]{Curryfication}
\begin{block}{Curryfication}
\begin{itemize}
 \item Une fonction ``curryfiée'' est invoquée avec un seul paramêtre, même si elle en attend plusieurs;
 \item Dans ce cas, elle renvoie une fonction qui attend les arguments manquant.
\end{itemize}
\end{block}
\begin{block}{Avec la STL}
\begin{lstlisting}
using namespace std::placeholders;
auto f = std::function<int(int,int,int)>([](int x, int y, int z) { return x*y+z; });
auto s = std::bind(f,4,_1,_2);  auto t = std::bind(s,5,_1);
auto w = t(6);
std::cout << "w = " << w << std::endl;
\end{lstlisting}
\end{block}
\begin{block}{Avec Boost Hana ( gratuit, template, compatible C++98, \ldots ) }
\begin{lstlisting}
using namespace boost::hana;
auto f = curry<3>([](int x, int y, int z) { return x*y + z; });
auto s = f(4); auto t = s(5);
auto w = t(6);
std::cout << "w = " << w << std::endl;
\end{lstlisting}
\end{block}
\end{frame}

\begin{frame}[fragile]{Fonction de mappage}
\tiny
\begin{block}{La fonction \lstinline$transform$}
 \begin{itemize}
  \item Tout langage fonctionnel a une fonction \lstinline$map$
  \item L'équivalent de la STL C++ est la fonction \lstinline$std::transform$.
\begin{lstlisting}
std::vector<int> v = {1,2,3,4};
std::transform(std::begin(v),std::end(v), std::begin(v), [](int x) { return x * 2; });
\end{lstlisting}
  \item Simple, rapide\ldots
  \item \alert{Mais si vous voulez changer le type d'éléments} ?
\begin{lstlisting}
std::vector<int> v = {1,2,3,4};
std::vector<std::string> outs;
std::transform(std::begin(v),std::end(v),std::back_inserter(outs), 
               static_cast<std::string(*)(int)>(std::to_string));
std::for_each(std::begin(outs),std::end(outs),[](std::string s) { std::cout << s << " "; });
std::cout << std::endl;
\end{lstlisting}  
  \item Gestion explicite de la taille du vecteur de sortie\ldots (\lstinline$std::back_inserter$);
  \item \alert{Puis je composer ces fonctions} ?
\begin{lstlisting}
std::vector<T1> input;
std::vector<T2> step_one;
std::vector<T3> output;
std::transform(std::begin(input),std::end(input),std::back_inserter(step_one), &f);
std::transform(std::begin(step_one),std::end(step_one),std::back_inserter(output), &g);
\end{lstlisting}
  \item Peu lisible et lent !
\end{itemize}
\end{block}
\end{frame}


\begin{frame}[fragile]{Utilitaires fonctionnels}
  \tiny
\begin{block}{Fonction objet surchargée ( uniquement avec hana )}
\begin{itemize}
 \item Fonction spécialisant ses actions selon le type tout en restant générique
\end{itemize}
\begin{lstlisting}
auto f = boost::hana::overload( [] (std::string s) { return s + s; },
                                [] ( auto       x) { return x * 2; });
std::vector<int> vint = {1,3,5,7,9,11};
std::vector<std::string> vstr {"Tin", "Ouaf", "Cou" };
std::transform(std::begin(vint), std::end(vint), std::begin(vint), f);
std::transform(std::begin(vstr), std::end(vstr), std::begin(vstr), f);
std::for_each(vint.begin(),vint.end(),[](int i) { std::cout << i << " "; });
std::cout << std::endl;
std::for_each(vstr.begin(),vstr.end(),[](std::string s) { std::cout << s << " "; });
std::cout << std::endl;                         
\end{lstlisting}
\end{block}
\begin{block}{Application partielle d'opérateurs binaires}
\begin{lstlisting}
using hana::_;
std::vector<int> v = {1,2,3};
std::transform(std::begin(v),std::end(v), std::begin(v), _ * 2);
std::for_each(v.begin(),v.end(),[](int i) { std::cout << i << " "; });
std::cout << std::endl;
\end{lstlisting}
\end{block}
\end{frame}

\begin{frame}[fragile]{Filtres et réductions}
\tiny
\begin{block}{filtres}
\begin{itemize}
\item Enlève des éléments d'un ensemble
\item Utilise \lstinline$std::remove_if$
\begin{lstlisting}
auto it= std::remove_if(vec.begin(),vec.end(), [](int i){ return !((i < 3) or (i > 8)) }); 
auto it2= std::remove_if(str.begin(),str.end(), [](string s){ return !(isupper(s[0])); });
\end{lstlisting}
\end{itemize}
\end{block}
\begin{block}{Réductions}
\begin{itemize}
\item Réduit un ensemble de valeur à une seule valeur en appliquant successivement un opérateur binaire;
\end{itemize}
\begin{lstlisting}
std::accumulate(vec.begin(),vec.end(),1, [](int a, int b){ return a*b; }); 
std::accumulate(str.begin(),str.end(),string(""),[](string a,string b){ return a+":"+b; });   
\end{lstlisting}
\end{block}
\end{frame}

\begin{frame}[fragile]{Fonctions pures et impures}
\tiny
\begin{block}{Caractéristiques des fonctions pures}
\begin{itemize}
\item Fonctions pures : Fonctions ne modifiant pas les valeurs de ses variables;
\item Fonctions impures: les autres fonctions.
\begin{tabular}{|c|c|}\hline
\rowcolor{blue}\textbf{Fonction pure} & \textbf{Fonction impure} \\ \hline
\rowcolor{cyan}Produit toujours même résultat avec les mêmes paramêtres & Peut produire des résultats différents avec même paramètres \\
\rowcolor{white}Aucun  effet de bord & Peut avoir des effets de bord \\
\rowcolor{cyan}N'altère jamais des états & peut altérer des états même globaux\\
\rowcolor{white}Facile à reordonner ou exécuter multithread & Analyse difficile de reordonnancement \\ \hline
\end{tabular}
\end{itemize}
\end{block}
\end{frame}

\begin{frame}[fragile]{Fonctionalités manquantes}
\tiny
\begin{block}{Évaluation paresseuse}
\begin{itemize}
\item N'évalue une expression que si nécessaire;
\item Exemple marchant en Haskell : 
\begin{lstlisting}[language=haskell]
length [ 1\2, 4-3, 2*3+1, 1/0, 7*3 ]
\end{lstlisting}
\item On peut faire de même en C++, mais uniquement à la compilation
\begin{lstlisting}
 template <typename... Args> void mySize(Args... args) { 
    cout << sizeof...(args) << endl; } 
mySize("Rainer",1/0);  \end{lstlisting}
\end{itemize}
\end{block}
\begin{block}{Compréhension de listes}
\begin{itemize}
\item Manque une fonction range donnant une étendue ( comme Haskell, Python, ...)
\begin{lstlisting}[language=python]
a = [ i for i in range(1000) if i%3==0 and i%5==0 ]
\end{lstlisting}
\item Prévu initialement pour C++17, mais reporté pour C++ 20;
\item Néanmoins, on peut utiliser Boost.Range ou range d'\'Eric Niebler ( prévu pour être la future norme )
\end{itemize}
\end{block}
\end{frame}

\begin{frame}[fragile]{Exercices}
\tiny
\begin{exampleblock}{Trie selon l'axe des abcisses}
\begin{itemize}
\item \'Ecrire un programme "minimaliste" permettant de trier des vecteurs de dimension 3 \lstinline$std::array<double,3>$ selon les abcisses;
\item Pour l'algorithme de tri, on regardera la fonction \lstinline$std::sort$ proposée par la STL ( dans \lstinline$algorithm$ ).
\end{itemize}
\end{exampleblock}

\begin{block}{Composition de fonctions et vecteur}
\begin{itemize}
\item Utiliser la composition de template vu dans ce chapitre pour construire des expressions arithmétiques pouvant être appliquées
sur chaque composante de vecteurs de dimension $N$.
\end{itemize}
\end{block}
\end{frame}

\end{document}